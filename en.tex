
\documentclass[11pt,a4paper]{moderncv}

\moderncvtheme[blue]{casual}

\definecolor{url}{rgb}{0.0, 0.44, 1.0}

\PassOptionsToPackage{colorlinks=true,urlcolor=url,pdfborderstyle={/S/U/W 1}}{hyperref}

\usepackage[utf8]{inputenc}
\usepackage[english]{babel}

\usepackage{amsmath}

\usepackage[scale=0.8]{geometry}
\AtBeginDocument{\recomputelengths}

\title{TechLead, software architect, developer, DevOps}
\name{Evgenii}{Lepikhin}
\address{Bishkek, Kyrgyzstan / Moscow, Russia}
\email{johnlepikhin@gmail.com}
\social[linkedin]{evgenii-lepikhin}
\social[github]{johnlepikhin}
\extrainfo{last update: Jan 2023}
\fax{+7(926)146-23-36 (Telegram, WhatsApp)}
\phone{+996 220 052 711}
\photo[64pt]{me.eps}

\PassOptionsToPackage{colorlinks=true}{hyperref}

\begin{document}

\makecvtitle

\section{In short}

\subsection{Experience}

\cventry{2017--present}{Automation team lead}{VK (ex. Mail.Ru)}{Design, development and implementation of dev environment for 400 engineers.
  Includes a new infrastructure for parallel builds and tests on a clusters of machines, workflow system for scheduling and robust
  execution.}{}{}

\cventry{2011--2017}{Senior DevOps, developer}{1Gb.ru}{Linux kernel development, antivirus software for JS, etc.}{}{}

\cventry{2003--2011}{Senior DevOps, team lead}{ISPsystem LLC}{High availability web hosting software development.}{}{}

\subsection{Strengths}

\cvline{Linux}{10+ years of experience at SRE level, thousands of hosts, participation in tasks from auto-deployment to writing kernel
  patches and building large scale systems. Networking, disk subsystem, IPC, securing, administration}

\cvline{Backend development}{Architecture planning, development management, preparation of the product for the realities of operation.}

\subsection{Recent key projects}

\cvline{2019--2023}{Development from scratch ``Platform'' dev environment: a distributed and fault-tolerant system of reproducible builds, centralized ACL integrated with the corporate database, dependency management, running reproducible autotests in a cluster environment and hosting for environments of colleagues.}
\cvline{2021--2023}{Developed a unified framework for securing external dependencies (gomods, rust crates, python pip, etc.) for builds and developer workstations.}
\cvline{2021--2022}{\href{https://github.com/mailru/shadowplay/}{Static analyzer} for Puppet SCM}

\subsection{Technologies}

\cvline{Current stack}{Linux, Rust, Puppet, C, C++, Java, Perl, Ocaml, Haskell, Elisp, Python, FreeBSD, Opensolaris}

\subsection{Miscellaneous}

\cvline{Degree}{high school}
\cvline{Languages}{English: intermediate, Russian: native}

\pagebreak

\section{In detail}

\subsection{Current field of interest}

Formal logic and software verification. I am enthusiastic about huge systems, refactoring, and building at the ecosystem level.

\subsection{Technology stack and strengths}

\cvline{Leading}{Pretty rich hiring experience: tons of interviews for developer, DevOps, and QA automation positions; created a system
  of partial automation and objectification of hiring. Successful experience in managing projects with a complexity of thousands of
  person-days. Passed a lot of courses from VK on management.}

\cvline{Paradigms}{Good understanding of functional, declarative, and OOP styles.}

\cvline{Async}{I can build cooperative multitasking from select () or epoll(), and then using promises (monads) create a DSL for more or
  less convenient work with the event processing FSM.}

\cvline{Computing}{I understand what map is and what reduce is, can assemble merge sort from them. I can distinguish hash from cache and
  passing from parsing by eye. I know which recursions have a tail (and why it is good in programming), I also know how to make it so that a
  non-tailed recursion finds it. I heard about Lamport's clock and distributed snapshots (in the context of studying
  \href{https://coq.inria.fr}{Coq} tried to prove the correctness of distributed snapshots based on Lamport's clock).}

\cvline{Language theory}{When writing parsers, DSLs, or learning new languages, I am well aware of the difference between a token, a
  grammatical unit, an expression, and semantics. I understand that for the ultimate goal, the quality of the AST is more important than the
  way brackets and indents are placed in a particular syntax.}

\cvline{Memory algorithms}{Familiar with some algorithms for memory hierarchies. For quick editing of a linked list, I can create a sparse
  array the size of a processor cache block. I understand why a loop over the values of a two-dimensional array in two ways will have
  different performance. I am familiar with slabs in the Linux kernel, I understand when they are needed and how to create an allocator. I
  know about memory barriers and why they are needed.}

\cvline{Compiles}{I am familiar with and created static analyzers, translators, interpreters. I have no experience in creating compilers to
  machine code.}

\cvline{Linux environment}{Continuous experience since 2000. I mainly use Debian-based and RH-based distributions. My laptop is running
  GuixSD. Automation of everything, deployment, operational search and problem-solving, infrastructure planning. Shell scripts, LAMP,
  \href{https://it.lepikhin.site/post/selinux-howto/}{SELinux}, glusterfs, DRBD and dozens of other technologies. In the distant past, I
  also actively worked with FreeBSD (everything except for the system kernel).}

\cvline{Linux kernel}{Developed patches for internal use (mainly minor improvements to OpenVZ, added functionality to procfs/sysfs to get
  statistics and control). I have some experience in developing modules. For one of the tasks, I digged deeply into the MM subsystem in
  order to find the causes of mysterious memory destruction.}

\cvline{VM}{Rich experience with qemu+kvm, OpenVZ and LXC. Administration of hosts with thousands of virtual machines in web hosting,
  created a specialized hosting for developers. I have some experience with the OpenStack Compute and libvirt frameworks.}

\renewcommand{\listitemsymbol}{-}

\pagebreak

\nocite{*}
\bibliographystyle{ieeetr}
\bibliography{en_publications}

\end{document}
