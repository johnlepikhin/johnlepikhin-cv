
\documentclass[11pt,a4paper]{moderncv}

\moderncvtheme[blue]{casual}

\definecolor{url}{rgb}{0.0, 0.44, 1.0}

\PassOptionsToPackage{colorlinks=true,urlcolor=url,pdfborderstyle={/S/U/W 1}}{hyperref}

\usepackage[utf8]{inputenc}
\usepackage[T1,T2A]{fontenc}
\usepackage[english,russian]{babel}

\usepackage{amsmath}

\usepackage[scale=0.8]{geometry}
\AtBeginDocument{\recomputelengths}

\title{TechLead, архитектор}
\name{Евгений}{Лепихин}
\address{Бишкек, Киргизия / Москва, РФ}
\email{johnlepikhin@gmail.com}
\social[linkedin]{evgenii-lepikhin}
\social[github]{johnlepikhin}
\extrainfo{последнее обновление: 23 января 2023}
\fax{+7(926)146-23-36 (Telegram, WhatsApp)}
\phone{+996 220 052 711}
\photo[64pt]{me.eps}

\begin{document}

\makecvtitle

\section{Кратко}

\subsection{Где работал}

\cventry{2017--н.в.}{Руководитель группы}{VK (Mail.Ru)}{Организация dev-среды для разработчиков, автоматизация тестирования бэкенда}{}{}

\cventry{2011--2017}{UNIX-специалист, программист}{1Gb.ru}{Разработка модулей в ядро Linux, антивируса для JS, ...}{}{}

\cventry{2003--2011}{UNIX-специалист, руководитель}{ISPsystem LLC}{Разработка софта для хостинга, архитектор кластерного хостинга}{}{}

\subsection{Что умею}

\cvline{Linux}{Опыт в IT SRE более 10 лет, тысячи машин, от автодеплоймента до написания патчей на ядро и построеня больших распределенных
  систем. Сеть, дисковая подсистема, IPC, безопасность, администрирование}

\cvline{Разработка бэкенда}{Планирование архитектуры, подготовка продукта к реалиям эксплуатации.}

\subsection{Последние ключевые проекты}

\cvline{2019--2023}{Разработка с нуля и успешный запуск <<Платформы>> разработчиков: распределенной и отказоустойчивой системы
  воспроизводимых сборок проектов, централизованного управления доступами интегрированного с корпоративной базой сотрудников, управления
  зависимостями, запуска воспроизводимых автотестов в кластерной среде, хостинга (liblxc) для рабочих окружений коллег. Сборка порядка 300
  проектов, до 500 сборок в сутки, около 400 пользователей.}

\cvline{2021--2023}{Создание единой инфраструктуры обеспечения безопрасности внешних зависимостей (gomods, rust crates, python pip и т.д.)
  для сборок и рабочих станций разработчиков.}

\cvline{2021--2022}{\href{https://github.com/mailru/shadowplay/}{Статический анализатор} для Puppet}

\subsection{Технологии}

\cvline{Ключевые навыки}{Linux на уровне SRE, senior разработчик Rust, DevOps, эксперт Puppet}

\cvline{Плотно работал}{C, C++, Java, Perl, Ocaml, Haskell, Elisp, Python, FreeBSD, Opensolaris, Grafana, Prometheus, NAS, DRDB, MySQL,
  PostgreSQL, технологии шардирования, естественные языки, парсинг ЯП}

\pagebreak

\section{В подробностях}

\subsection{Текущие интересы}

Формальная логика и системы доказательств корректности кода. С воодушевлением отношусь к большим системам, рефакторингу и строительству на
уровне экосистем.

\subsection{Доступные инструменты и технологии}

\cvline{Менеджмент}{Довольно богатый опыт найма: сотни собеседований на позиции разработчика, девопса и автотестера; создал систему
  частичной автоматизации и объективизации найма. Успешный опыт ведения проектов сложностью в тысячи человеко-дней. Проходил множество
  курстов от ВК по менеджменту. Agile, спринты, покер планирования, GTD, риск-менеджмент, выращивание специалистов, мотивация, гант,
  роадмапы, канбан, приоритетизация задач.}

\cvline{Парадигмы}{Прекрасно понимаю функциональный стиль, неплохо знаком с ООП. Могу писать императивный лапша-код за 5 минут, но
  предпочитаю потратить час и написать код, соответствующий моему представлению о прекрасном.}

\cvline{Async}{Смогу из select() или epoll/kqueue соорудить кооперативную многозадачность, а потом с использованием promises
  написать DSL для более-менее удобной работы с конечным автоматом обработки событий.}

\cvline{Вычисления}{Понимаю что такое map и что такое reduce, смогу из них собрать merge sort. Умею на глаз отличать хэш от кэша и пассинг
  от парсинга. Знаю у каких рекурсий бывает хвост (и чем это хорошо в программировании), знаю как сделать так, чтобы не хвостатая рекурсия
  обрела его. Что-то слышал про часы Лэмпорта и распределенные снапшоты (в контексте изучения \href{https://coq.inria.fr}{Coq} доказывал
  корректность distributed snapshots).}

\cvline{Теория языков}{При написании парсеров, DSL или изучении новых языков хорошо осознаю разницу между токеном, лексемой, грамматической
  единицей, выражением и семантикой. Понимаю, что для конечной цели важнее качество AST, чем способ расстановки скобочек и отступов в
  конкретном синтаксисе.}

\cvline{Память и алгоритмы}{Имею некоторое знакомство с алгоритмами для иерархий памяти. Для быстрого редактирования linked list смогу
  изобразить разреженный массив размером с блок кэша процессора. Понимаю, почему при проходе значений двумерного массива двумя способами и
  будут сильно разные скорости. Видел код slab'ов в ядре Linux, понимаю когда они нужны и как сделать свой аллокатор. Знаю про memory
  barriers и о том, как мы до этого докатились.}

\cvline{Компиляторы}{Знаю и делал на практике статические анализаторы, трансляторы и интерпретаторы языков.}

\cvline{Анализ данных}{Не силён. Могу написать элементарный поиск аномалий в timeline с использованием скользящего среднего или подобных
  простых техник, могу с нуля реализовать под эту задачу нейросеть в одной из простых архитектур, типа трех слоев с обратным
  распространением ошибки (подозреваю, что L2TP будет лучше, но экспертиза недостаточна). Осилю написать с нуля наизусть свертку матрицы
  значений в вектор признаков. Упущение: никогда не программировал для GPU.}

\subsection{Знакомство с системами}

\cvline{Окружение Linux}{Непрерывный опыт с 2000 года. Дистрибутивы основном Debian-based и RH-based. Рабочая станция на GuixSD.
  Автоматизация чего угодно, деплоймент, оперативный поиск и решение проблем, планирование инфраструктуры. Шелл-скрипты, LAMP,
  \href{https://it.lepikhin.site/post/selinux-howto/}{SELinux}, glusterfs, DRBD и десятки других технологий. В далеком прошлом также активно
  работал с FreeBSD (со всем, за исключением ядра системы).}

\cvline{Ядро Linux}{Писал патчи для внутреннего использования (в основном мелкие доработки OpenVZ, дописывание ручек для получения
  статистики и управления через procfs/sysfs, небольшие оптимизации), есть небольшой опыт разработки модулей. Есть опыт глубокого погружения
  в mm с целью нахождения причин загадочных разрушений памяти и путей оптимизации.}

\cvline{VM}{Богатый опыт с qemu+kvm и openvz. Администрирование хостов с тысячами вируальных машин в хостингах, создание своего
  специализированного хостинга для разработчиков. Есть некоторый опыт с публичными фреймворками: OpenStack Compute и libvirt.}

\section{Ожидания}

\cvline{От коллег}{Хотя бы в общих чертах сформулированное ТЗ и путей дальнейшего движения проекта. На слух могу многое упустить, почта или
  чаты~--- наше всё.}

\cvline{От организации времени}{Свободный график, возможность раз в 1-2 года брать неоплачиваемый отпуск на 20-30 дней: занимаюсь
  экспедиционным альпинизмом и 28 дней на поездку обычно не хватает.}

\subsection{Прочее}

\cvline{Иностранные языки}{Английский: intermediate}
\cvline{В жизни}{1982 г.р. Женат, двое детей. Проживаю за пределами РФ.}


\renewcommand{\listitemsymbol}{-}

\pagebreak

\nocite{*}
\bibliographystyle{plain}
\bibliography{ru_publications}

\end{document}
